% ------------------------------------------------------------------- %
% Title

\newpage
\thispagestyle{empty}

\vspace{3.0cm}

\bigskip
\centerline{ \Large{ \textbf{ MORIS }}} 
\vspace{0.3cm}
\centerline{ \Large{ \textbf{ Mesh Generation }}}

% ------------------------------------------------------------------- %
% Date

\vspace{1.0cm}

\begin{center}
	\centerline{Last updated:} 
    \today
\end{center}


% ------------------------------------------------------------------- %
% Abstract

\vspace{1.6cm}

\paragraph{Preface}
This document serves as a user guide to using MORIS' mesh generation abilities to output body-fitted finite element meshes alongside extraction operators projecting the basis of the foreground mesh into a tensor-product B-spline basis. The mesh data generated is intended for \emph{interpolation-based immersed finite element analysis} as introduced by Fromm et al. \cite{Fromm2022}. 

The document is split into two parts. The first part, \Cref{sec:overview},  provides an overview of some theoretical and algorithmic aspects, the second part, \Cref{sec:tutorial}, lays out all currently supported features and how to use them with an \textsc{XML} input file. 

This guide assumes that MORIS has already been installed and compiled. An installation guide is provided in the MORIS GitHub repository:
\href{https://github.com/kkmaute/moris/tree/main/share/install}{github.com/kkmaute/moris}.
Note that the features outlined in this document currently undergo expansion and changes. Please check back regularly for any updates.

If you have questions, comments, feature requests, or find any bugs, please contact \href{mailto:nils.wunsch@colorado.com}{Nils Wunsch \Letter}


% ------------------------------------------------------------------- %
% Authorship

\vspace{1.8cm}

\centerline{This documentation is part of MORIS which is licensed under the MIT license. }
\centerline{See \href{https://github.com/kkmaute/moris/blob/main/LICENSE.txt}{LICENSE.txt \ExternalLink} for details.}

\vspace{1.5cm}

\centerline{Topology Optimiation Research Group}
\centerline{Aerospace Mechanics Research Center (AMReC)}
\centerline{Ann \& H. J. Smead Department of Aerospace Engineering Sciences}
\centerline{University of Colorado Boulder}


% ------------------------------------------------------------------- %
\newpage
