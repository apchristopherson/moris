% Settings for english language
% last language is used language
\usepackage[english]{babel}

% Font Encoding (Ausgabe); T1 contains Umlauts as separate glyphs
\usepackage[T1]{fontenc}

% enable pdf_tex files to be used for figures
\usepackage{import}

% Input file encoding; Save all files as utf8 in your editor! (exception: BibTeX-files)
\usepackage[utf8]{inputenc}

% Type1-font in pdf-file (Latin Modern)
\usepackage{lmodern} 

% Set margins
\usepackage{textcomp}

% automatically set white spaces for consistent formatting
\usepackage{xspace}

% set custom line spacing 
\usepackage{setspace}

% beautify justified text
\usepackage{microtype}

% dropped capital letter at beginning of paragraph
\usepackage{lettrine}
           
% enable include of whole pdf pages
\usepackage{pdfpages}

% add margin notes to document: \marginpar{text} --> text is printed in the right or outer margin of the page
\usepackage{marginnote}    

% allow for single landscape pages
\usepackage{lscape}

% Utilities for figures
\usepackage{graphicx} 
\usepackage{transparent}
\usepackage{subcaption}
\usepackage{float}                % enables dynamic placing of figures
% \usepackage{wrapfig}            % enables figures in textblocks
\usepackage{color} 
\usepackage{geometry}
\usepackage{rotating}
\usepackage{tikz}
\graphicspath{{./Figures/}}

% enable use of lists with description items
\usepackage{enumitem}   

% text symbols
\usepackage{marvosym}

% use listings and define color of comments etc. and format
\usepackage{listings}
\usepackage{color}
  
%%%%%%%%%%%%%%%%%%
%% Typeset Math %%
%%%%%%%%%%%%%%%%%%

% math formatting
\usepackage{bm}
\usepackage{upgreek}
\usepackage{booktabs}
\usepackage{multirow}

% Intlimits setzt Integralgrenzen ober-/unterhalb des Integralzeichens und nicht seitlich versetzt
\usepackage[intlimits]{amsmath}

% amsthm:     extends Latex's \newtheorem command
% amssymb:    enables the use of special symbols
\usepackage{amsthm, amssymb}

% colored tables
\usepackage{colortbl}

% Streichen
\usepackage{cancel}

% beautifies arrays
\usepackage{esvect}

% SI-Units
% locale = US --> 1.2 m
% per-mode=fraction --> Bruchstriche
%\usepackage[locale = US,per-mode=fraction]{siunitx}
\usepackage{siunitx}
%Kopf- und Fußzeile
\usepackage{scrlayer-scrpage}  
\usepackage{scrhack}

% ifthen:    if/then decision
\usepackage{ifthen}

% FixME: für Hinweise in der draft-Version, aber nicht in der Final-Version... (notes on things that should be fixed)
% \usepackage{fixme}

% filler texts
\usepackage{lipsum}            

% Setup fixme package
% layout = inline     --> display note inline ??
% theme = color     --> using colors
% silent = true        --> suppress logging (notes are not recorded in the log file)
% lang = 
% \fxsetup{ layout=inline,
% theme=color,
% silent=true,
% lang=ngerman}

% Korrektur der \listoffixmes zur Verwendung mit KOMA-Skript-Klassen
% makeatletter:    weist at-Zeichen Kategorie-Code 11 zu (11 = normale Zeichen/Klein- und Großbuchstaben)
\makeatletter

% toc = table of contents
% lox = Dateiendung
%\addtotoclist[float]{lox}
% setuptoc{Dateiendung}{Liste von Eigenschaften}
% chapteratlist:    sorgt dafür, dass in dieses Verzeichnis bei jedem neuen Kapitel eine optionale Gliederung eingefügt wird
%\setuptoc{lox}{chapteratlist}
% Überschriften
%\renewcommand*{\lox@heading}{\listoftoc[{\@fxlistfixmename}]{lox}}
% makeatother: weist at-Zeichen Kategorie-Code 12 zu (12 = andere Zeichen(at war ursprünglich Kategorie-Code 12))
\makeatother
% lot = list of tables?
%\setuptoc{lot}{totoc}


% Letztes package: Hyperreferenzierung    
\definecolor{webblack}{rgb}{0,0,0} %black
\usepackage{hyperref}
\hypersetup{
    unicode=true,           % non-Latin characters in Acrobat’s bookmarks
    pdftoolbar=true,        % show Acrobat’s toolbar?
    pdfmenubar=true,        % show Acrobat’s menu?
    pdffitwindow=false,     % window fit to page when opened
    pdfstartview={FitH},    % fits the width of the page to the window
    pdfnewwindow=true,      % links in new window
    colorlinks=false,       % false: boxed links; true: colored links
    linkcolor=red,          % color of internal links (change box color with linkbordercolor)
    linktoc=all,                        % defines which part of an entry in the table of contents is made into a link
    citecolor=green,        % color of links to bibliography
    filecolor=magenta,      % color of file links
    urlcolor=cyan,          % color of external links
    bookmarksopenlevel=1,
    bookmarksopen=true
}

\usepackage[capitalise, nameinlink]{cleveref}

\usepackage[all]{hypcap}

% WICHTIG: Ab hier kein usepackage mehr !!!
